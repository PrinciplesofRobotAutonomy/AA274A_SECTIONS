\section{The Turtlebot Software}
Most of the forward-facing Turtlebot software you will work with is located in the \texttt{asl\_turtlebot} repository on Github. To get it, go to \texttt{\textasciitilde/catkin\_ws/src} on the provided laptop and if the \texttt{asl\_turtlebot} repository doesn't already exist, run:

\begin{lstlisting}
# run this command on the laptop
git clone https://github.com/StanfordASL/asl_turtlebot.git
\end{lstlisting}

or if the repository does exist, then \texttt{cd} into the directory and run:

\begin{lstlisting}
# run this command on the laptop
git pull
\end{lstlisting}

to update to the latest code.

Since we downloaded a new catkin package, we need to rebuild the workspace by running the following from the \texttt{\textasciitilde/catkin\_ws} directory.

\begin{lstlisting}
# run this command on the laptop
catkin_make
\end{lstlisting}

\subsection{Turtlebot bring up}
First, we must take some steps to configure the laptop in order to be able to connect to a TurtleBot. You will see $rostb3.sh$ and $roslocal.sh$ in the \texttt{asl\_turtlebot} folder. These files are important for telling your computer where roscore lives. %Take a look at the files. Can you figure out roughly what is it doing? 
Specifically, for the laptop to communicate (send/receive messages) with the robot, it needs to know the network address of the robot. To do so, three environment variables are important: \texttt{ROS\_MASTER\_URI}, \texttt{ROS\_HOSTNAME}, and \texttt{ROS\_IP}.


We will now set up these scripts so it's easy to switch between them.
\begin{enumerate}
	% \item Copy those files in your home directory.
	\item Connect to the correct network. (The TA will tell you which one it is.)
	\item Edit \texttt{rostb3.sh} accordingly: define \texttt{TURTLEBOT\_NAME} at the start of the script.
	\item Source \texttt{rostb3.sh}.
	\begin{lstlisting}
# run this command on the laptop
source rostb3.sh
	\end{lstlisting}
	\item Open your \texttt{.bashrc} with a text editor. All the shell commands in this file will get run whenever you open a terminal.
	Add the following lines to the end of the file:
	\begin{lstlisting}
# run these commands on the laptop
alias rostb3='source ~/catkin_ws/src/asl_turtlebot/rostb3.sh'
alias roslocal='source ~/catkin_ws/src/asl_turtlebot/roslocal.sh'
export TURTLEBOT3_MODEL=burger
	\end{lstlisting}
	The \texttt{TURTLETBOT3\_MODEL} should remain \texttt{burger}, do not change this.
	IMPORTANT: This will create an alias for rostb3 and roslocal. If roscore is to run on the TurtleBot, and you want to run nodes from your computer (not ssh), you must type rostb3 EVERY TIME you open a terminal window. Otherwise, if you want to run things locally on your machine, you should run roslocal.
	\item For these modifications to take effect in the current terminal, run:
	\begin{lstlisting}
# run this command on the laptop
source ~/.bashrc
	\end{lstlisting}
\end{enumerate}

Next, in a terminal window, ssh into the TurtleBot using:
\begin{lstlisting}
# run this command on the laptop
ssh aa274@<TurtleBot Name>.local
\end{lstlisting}

with the password \texttt{aa274}. You don't need the angle brackets ($<>$). This remotely logs into the onboard robot computer. The necessary ROS packages
and drivers for TurtleBot operation have been pre-installed so we can go ahead and run:

\begin{lstlisting}
# run this command on the robot
roslaunch turtlebot3_bringup turtlebot3_core.launch	
\end{lstlisting}

to launch core packages to start up the TurtleBot.

{\bf Problem 1: Once this is all running, which rostopics are available? Paste this list in your submission.}
