\section{TurtleBot Teleoperation}

Now, let's explore teleoperation with the TurtleBot.

\begin{enumerate}
\item We can start exploring the existing ROS topics. What are all the messages that are being published right now? In particular, look at the odom topic. What is the message type being published to this topic and what information is contained within these messages? HINT: rostopic info odom might help.
\item In a new terminal window, begin teleoperating the robot by running:
\begin{lstlisting}
roslaunch turtlebot3_teleop turtlebot3_teleop_key.launch
\end{lstlisting}
\item Try to teleop the TurtleBot back to $(0,0,0)$.
\end{enumerate}

{\bf Problem 2: What is the message type being published to \texttt{odom} and what information is contained within these messages?}

\subsection{Pub to cmd\_vel}
Using our code from last week's section, create a publisher that publishes to the \texttt{cmd\_vel} topic and sends a zero velocity signal at every timestep. The skeleton code for this included in this week's code in the \texttt{vel\_publisher.py} file. In particular, you should send out a message of type \texttt{geometry\_msgs/Twist}, with information for how to populate it available online. Some resources that help are the \href{http://docs.ros.org/melodic/api/geometry_msgs/html/msg/Twist.html}{ROS documentation on it} as well as \href{https://github.com/StanfordASL/asl_turtlebot/blob/master/scripts/keyboard_teleop.py}{our own TurtleBot code} (look at line 155).

{\bf Problem 3: Paste your code in your submission, as well as any of its running output.}

\subsection{Sub to odom}
Similarly, create a subscriber that subscribes to the odom topic and prints out what it receives. The skeleton code for this is located in the \texttt{odometry\_subscriber.py} file.

{\bf Problem 4: Paste your code in your submission, as well as any of its running output.}